\documentclass[a4paper]{article}
\usepackage[margin=1in]{geometry}
\usepackage[utf8]{inputenc}
\usepackage[english]{babel}
%\usepackage{graphicx}
\usepackage{amssymb}
\usepackage{amsmath}
%\usepackage[thmmarks,  thref]{ntheorem}
\usepackage{amsthm}
%\usepackage{listings} % Darstellen von Codebeispielen
\usepackage{titlesec}
%\usepackage{chngcntr}
\usepackage{mathtools}
\usepackage{algorithm}
\usepackage[noend]{algpseudocode}
\usepackage[normalem]{ulem}
%\usepackage{float}
%\usepackage{cite}
\counterwithin{figure}{section}
\newcommand{\sectionbreak}{\clearpage}
\usepackage[round]{natbib}
\usepackage{float}
\usepackage{svg}
\usepackage{fancyhdr}
\pagestyle{fancy}
\usepackage[hidelinks]{hyperref}
\usepackage[textsize=tiny]{todonotes}

\mathtoolsset{showonlyrefs}

\DeclareMathOperator{\argmax}{argmax}
\DeclareMathOperator{\argmin}{argmin}
%\DeclareMathOperator{\next}{next}
\DeclareMathOperator{\prev}{prev}
\DeclareMathOperator{\on}{on}
\DeclareMathOperator{\off}{off}
\DeclareMathOperator{\idle}{idle}
\DeclareMathOperator{\act}{active}
\DeclareMathOperator{\costs}{costs}
\DeclareMathOperator{\adjacent}{adjacent}
\DeclareMathOperator{\neighboring}{neighboring}
\DeclareMathOperator{\true}{true}
\DeclareMathOperator{\false}{false}
\DeclareMathOperator{\ifop}{if}
\DeclareMathOperator{\elseop}{otherwise}
\DeclareMathOperator{\OPT}{OPT}
\DeclareMathOperator{\PLTR}{pltr}
\DeclareMathOperator{\LTR}{ltr}
\DeclareMathOperator{\aug}{aug}
\DeclareMathOperator{\real}{real}
\DeclareMathOperator{\fv}{fv}
\DeclareMathOperator{\uv}{uv}
\DeclareMathOperator{\pv}{pv}
\DeclareMathOperator{\vol}{v}
\DeclareMathOperator{\opdef}{def}
\DeclareMathOperator{\exc}{exc}
\DeclareMathOperator{\keepidle}{keepIdle}
\DeclareMathOperator{\keepactive}{keepActive}
\DeclareMathOperator{\rank}{rank}
\DeclareMathOperator{\crit}{crit}
\DeclareMathOperator{\spac}{space}
\DeclareMathOperator{\fillop}{fill}
\DeclareMathOperator{\close}{close}
\DeclareMathOperator{\res}{Res}

\newtheorem{theorem}{Theorem}
\newtheorem{lemma}[theorem]{Lemma}
\newtheorem{definition}[theorem]{Definition}

%\theoremheaderfont{\itshape}
%\theorembodyfont{\upshape}
% wtf is this?
%\newtheoremstyle{case}{}{}{}{}{}{:}{ }{}
%\theoremstyle{case}
%\newtheorem{case}{Case}

%\newtheorem*{remark}{Remark}


\title{Greedy Energy-Efficient Scheduling Algorithms for Processor Systems}
%\author{Gunther Bidlingmaier}
%\date{15.11.2020}

\begin{document}

\selectlanguage{english}
%\frontmatter{}
%\input{pages/acknowledgements}

%\maketitle
\begin{abstract}
  We study a particular scheduling setting in which a set of $n$ jobs with individual release times and deadlines has to be scheduled across $m$ homogeneous processors while minimizing the consumed energy.
  Idle processors can be turned off so as to save energy, while turning them on requires a fixed amount of energy.
  For the special case of a single processor, the greedy algorithm Left-to-Right guarantees an approximation factor of $2$.
  We generalize this simple greedy policy to the case of multiple processors and show that the energy costs are still bounded by $2 \OPT + P$.
  Our algorithm has a running time of $(n + m) \log(d^*) F$ where $d^*$ is the largest deadline and $F$ the costs of a maximum flow calculation for checking feasibility of an instance.
\end{abstract}

\tableofcontents

\section{Algorithm}
\todo{Formal Problem definition, notation}
\todo{Formulate and format keepactive, keepidle nicer.}
\todo{Properly define scheduling problem with lower and upper bounds $l_t, m_t$.}
\todo{Precisely relate $l_t, m_t$ returned by Algorithm to assignment of jobs to processors, time slots.}
\begin{algorithm}[H]
\caption{Parallel Left-to-Right}\label{alg:pltr}
\begin{algorithmic}
  \State{} $m_t \gets m$
  \State{} $l_t \gets 0$
  \For{$k \gets m$ \textrm{to} $1$}
    \State{$t\gets0$}
    \While{$t < d^*$}
      \State{$t \gets $\Call{KeepIdle}{k, t}}
      \State{$t \gets $\Call{KeepActive}{k, t}}
    \EndWhile{}
  \EndFor{}

  \Function{KeepIdle}{k, t}
    \State{search for maximal $t' \geq t$ s.t.\
    exists feasible schedule with $m_{t''} = k-1 \forall t'' \in [t, t')$}
    \State{$m_{t''} \gets k - 1 \forall t'' \in [t, t')$}
    \State{\Return{$t'$}}
  \EndFunction{}
  \Function{KeepActive}{k, t}
    \State{search for maximal $t' \geq t$ s.t.\
    exists feasible schedule with $l'_{t''} = \max\{k, l_{t''}\}k-1 \forall t'' \in [t, t')$}
    \State{$m_{t''} \gets k - 1 \forall t'' \in [t, t')$}
    \State{\Return{$t'$}}
  \EndFunction{}
  %\Function{IsFeasible}{\null}
  %  \State{$f^* \gets$ Max-Flow for current values of $m_t, l_t$}
  %  \State{\Return{$f^* == P$}}
  %\EndFunction{}
\end{algorithmic}
\end{algorithm}


\begin{figure}[H]
  \centering
  \includegraphics[width=\textwidth]{graphics/sketches/flow_network.png}\label{fig:flow}
  \caption{The Flow-Network for checking feasibility of a scheduling instance with lower and upper bounds $l_t$ and $m_t$ for the number of active processors at $t$.}
\end{figure}

\todo{Ambiguity of $t$ used for time slots and sink in flow-network: Use $\alpha$-$\omega$ flow}
\todo{Briefly define $c$ for edge capacities and cuts.}
\todo{Introduce notation for number of jobs scheduled at $t$?}

\begin{lemma}\label{lemma:flow_feasibility}
  There exists a feasible solution to a scheduling instance with lower and upper bounds $l_t, m_t$ if and only if the maximum $s$-$t$ flow in the corresponding flow network depicted in Figure~\ref{fig:flow} has value $P$.
\end{lemma}
\begin{proof}
  Let $f$ be a $s$-$t$ flow of value $|f| = P$.
  We construct a feasbile schedule from $f$ respecting the lower and upper bounds given by $l_t$ and $m_t$.
  For every $j \in J$ and  $t \in T$, if $f(u_j, v_t) = 1$, then schedule $j$ at slot $t$.
  Since $|f| = P$ and the capacity $c(\{s\}, V \setminus \{s\}) = P$, we have $f_{in}(u_j) = p_j$ for every $j \in J$.
  Hence $f_{out}(u_j) = \sum_{t \in E_j} f_{in}(v_t) = p_j$.
  Hence every job $j$ is scheduled in $p_j$ distinct time slots.

  The schedule respects the upper bounds $m_t$, since $c(v_t, w) + c(v_t, t) \leq m_t - l_t + l_t$ and for every $t$ at most $m_t$ jobs are scheduled at $t$.

  The schedule respects the lower bounds $l_t$, since
  $c(V \setminus \{t\}, \{t\}) = P$ and hence
  $f(v_t, t) = l_t$ for every slot $t$.
  By flow conservation we then have $f_{in}(v_t) \geq l_t$ which implies that at least $l_t$ jobs are scheduled at every slot $t$.

  For the other direction consider a feasible schedule respecting the lower and upper bounds $l_t, m_t$.
  We construct a flow $f$ of value $P$ and show that it is maximal.

  If $j$ is scheduled at slot $t$ and hence $t \in E_j$,
  define $f(u_j, v_t) = 1$, otherwise $f(u_j, v_t) = 0$.
  Define $f(s, u_j) = p_j$ for every $j \in J$.
  Hence we have $f_{in}(u_j) = p_j$
  and $f_{out}(u_j)$ must be  $p_j$ since this corresponds to the number of distinct time slots in which $j$ is scheduled.
  Define $f(v_t, t) = l_t$ for every slot $t$.
  Define $f(v_t, w) = f_{in}(v_t) - l_t$.
  We have $f(v_t, w) \leq m_t - l_t$ since $f_{in}(v_t)$ corresponds to the number of jobs scheduled at $t$, which is at most $m_t$.
  We also have $f_{out}(v_t) = f_{in}(v_t) - l_t + l_t = f_{in}(v_t)$.

  Define $f(w, t) = P - \sum_t l_t$.
  Then $f_{in}(w) = \sum_t f_{in}(v_t) - l_t
  = \sum_t |\{j \in J \mid j~\text{scheduled at}~t\}| - \sum_t l_t$.
  Since the schedule is feasible, this corresponds to
  $f_{in}(w) = P - \sum_t l_t = f_{out}(w)$.

\end{proof}

\section{Structure of the PLTR-Schedule}
\subsection{Preliminary Definitions}
\todo{Use $T = T$ here?}
\begin{definition}
  For schedule $S$, we define the \emph{volume} $\vol_S(j, Q)$ of job $j \in J$ in a set $Q \subseteq T$ of time slots as the number of time slots of $Q$ for which $j$ is scheduled at by $S$.
\end{definition}
\begin{definition}
  We define the \emph{forced volume} $\fv(j, Q)$ of job $j \in J$ in a set $Q \subseteq T$ of time slots as the number of time slots of $Q$ for which $j$ has to be scheduled in every feasible schedule, i.e.
  \begin{align}
    \fv(j, Q) = \max\{0; p_j - |E_j \setminus Q|\} \text{.}
  \end{align}
\end{definition}
\begin{definition}
  We define the \emph{unnecessary volume} $\uv_S(j, Q)$ of job $j \in J$ in a set $Q \subseteq T$ of time slots as the amount of volume which does not have to scheduled during $Q$, i.e.
  \begin{align}
    \uv_S(j, Q) = \vol_S(j, Q) - \fv(j,Q)\text{.}
  \end{align}
\end{definition}
\begin{definition}
  We define the \emph{possible volume} $\pv(j, Q)$ of job $j \in J$ in a set $Q \subseteq T$ of time slots as the maximum amount of volume which $j$ can be feasibly scheduled in $Q$, i.e.
  \begin{align}
    \pv(j, Q) = \min\{ p_j, | E_j \cap Q | \} \text{.}
  \end{align}
\end{definition}
\begin{definition}
  We define the \emph{space $\spac_S(j, Q)$ of job $j \in J$ in a set $Q \subseteq T$ for schedule $S$} as the number of additional time slots, which $j$ can be scheduled in $Q$, i.e.\
  \begin{align}
    \spac_S(j, Q) = \pv(j, Q) - \vol_S(j, Q) \text{.}
  \end{align}
\end{definition}
Since the corresponding schedule $S$ will always be clear from context, we drop the subscript for $\vol, \uv$ and $\spac$.
We extend our volume definitions to sets $J' \subseteq J$ of jobs by summing over all $j \in J'$, i.e.
\begin{align}
  \vol(J', Q) = \sum_{j \in J'} \vol(j, Q)\text{.}
\end{align}
If the first parameter is dropped, we refer to the whole set $J$, i.e.\ $\vol(Q) = \vol(J, Q)$.
Clearly we have for every feasible schedule, every $Q, j$ that $\fv(j, Q) \leq \vol(j, Q) \leq \pv(j, Q)$.

\begin{definition}
  We define the \emph{density} $\phi(Q)$ for a set $Q \subseteq T$ as the average amount of processing volume which has to be completed in every slot of $Q$, i.e.
  $\phi(Q) = \fv(J, Q) / |Q|$.
  We also define $\hat \phi(Q) = \max_{Q' \subseteq Q} \phi(Q')$.
\end{definition}
If $\hat \phi(Q) > k - 1$, then clearly at least $k$ processors are required in some time slot $t \in Q$ for every feasible schedule.
\begin{definition}
  We define the \emph{deficiency} $\opdef(Q)$ of a set $Q \subseteq T$ of time slots as the difference between the amount of volume which has to be completed in $Q$ and the processing capacity available in $Q$, i.e. $\opdef(Q) = \fv(Q) - \sum_{t \in Q} m_t$.
\end{definition}
\begin{definition}
  We define the \emph{excess} $\exc(Q)$ of a set $Q \subseteq T$ of time slots as the difference between the processor utilization required in $Q$ and the amount of processing volume available in $Q$, i.e. $\exc(Q) = \sum_{t \in Q} l_t - \pv(Q)$.
\end{definition}
\subsection{Critical set of time slots}
\begin{lemma}\label{lemma:cut}
  For every $s$-$t$ cut $(S, \bar S)$ we have at least one of the following two lower bounds for the capacity $c(S)$ of the cut:
  $c(S) \geq P - \opdef(Q(S))$ or $c(S) \geq P - \exc(Q(\bar S))$, where $Q(S) \coloneqq \{ t \mid v_t \in S \}$.
\end{lemma}
\begin{proof}
  Let $(S, \bar S)$ be a $s$-$t$ cut, let $J(S) \coloneqq \{j \mid u_j \in S\}$.
  If $w \notin S$, consider the contribution of every node of $S$ to the capacity of the cut.
  \begin{itemize}
    \item Node $s$: $\sum_{j \in J(\bar S)} p_j$.
    \item Node $u_j$: $|\{v_t \in \bar S \mid t \in E_j\}| = | E_j \setminus Q(S) | \geq p_j - \fv(j, Q(S))$
    \item Node $v_t$: $l_t + m_t - l_t = m_t$
  \end{itemize}
  The inequality for node $u_j$ follows since $\fv(j, Q(S)) = \max \{0, p_j - |E_j \setminus Q(S)| \}$.
  In total, we can lower bound the capacity with
  \begin{align}
    c(S) &\geq \sum_{j \in J(\bar S)} p_j + \sum_{j \in J(S)} p_j - \fv(j, Q(S)) + \sum_{t \in Q(S)} m_t
    \\ &= P - \fv(J(S), Q(S)) + \sum_{t \in Q(S)} m_t
    \\ &\geq P - \opdef(Q(S))\text{.}
  \end{align}
  If $w \in S$, again consider the contribution of every node of $S$ to the capacity of the cut.
  \begin{itemize}
    \item Node $s$: $\sum_{j \in J(\bar S)} p_j \geq \pv(J(\bar S), Q(\bar S))$.
    \item Node $u_j$: $| E_j \setminus Q(S) |
      = | E_j \cap Q(\bar S)| \geq \pv(j, Q(\bar S))$
    \item Node $v_t$: $l_t$
    \item Node $w$: $P - \sum_t l_t$
  \end{itemize}
  In total, we can lower bound the capacity with
  \begin{align}
    c(S) &\geq P - \sum_{t \in Q(\bar S)} l_t + \pv(Q(\bar S))
    \\ &= P - \exc(Q(\bar S))
  \end{align}

\end{proof}

\begin{lemma}\label{lemma:feasibility}
  A scheduling instance with lower and upper bounds $l_t$ and $m_t$ is feasible if and only if $\opdef(Q) \leq 0$ and $\exc(Q) \leq 0$ for every $Q \subseteq T$.
\end{lemma}
\begin{proof}
  If $\opdef(Q) > 0$ for some $Q$, then some upper bounds $m_t$ cannot be met.
  If $\exc(Q) > 0$ for some $Q$, then some lower bound $l_t$ cannot be met.
  For the direction from right to left, consider an infeasible scheduling instance with lower and upper bounds.
  By Lemma~\ref{lemma:flow_feasibility} we have that the maximum flow $f$ for this instance has value $|f| < P$.
  Hence, there must be a $s$-$t$ cut $(S, \bar S)$ of capacity $c(S) < P$.
  Lemma~\ref{lemma:cut} now implies that $\opdef(Q(S)) > 0$ or $\exc(Q(\bar S)) > 0$.
\end{proof}
\begin{lemma}\label{lemma:critical}
  For every time slot $t \in T$ for which some processor $k \in [m]$ is activated in $S_{\PLTR}$, there exists a set $Q \subseteq T$ of time slots with $t \in Q, $
  \begin{align}
    \fv(Q) &= v(Q) \text{,}\\
    \vol(t') &\geq k-1 &~\text{for}~t' \in Q~\text{and}\\
    \vol(t') &\geq k &~\text{for}~t' \in Q~\text{with}~t' \geq t \text{.}
  \end{align}
\end{lemma}
\begin{proof}
  Suppose for contradiction there is some activation $t \in T$ of processor $k \in [m]$ and no such $Q$ exists for $t$.
  We show that $\PLTR$ would have extended the idle interval on processor $k$ which ends at $t$.
  Consider the step in $\PLTR$ when $t$ was the result of $\keepidle$ on processor $k$ and the corresponding lower and upper bounds $m_{t'}, l_{t'}$ for $t' \in T$ right after the calculation of $t$ and the corresponding update of the bounds by $\keepidle$.
  We modify the bounds by decreasing $m_t$ by $1$.
  Note that at this point $m_{t'} \geq k$ for every $t' > t$ and $m_{t'} \geq k - 1$ for every $t'$.

  Consider $Q \subseteq T$ s.t.\ $t \in Q$ and $\fv(Q) < \vol(Q)$.
  Before our modification we had $m_Q \coloneqq \sum_{t' \in Q} m_{t'} \geq \vol(Q) > \fv(Q)$.
  The inequality $m_Q \geq \vol(Q)$ here follows since the upper bounds $m_{t'}$ are monotonically decreasing during $\PLTR$.
  After our modification we still have $m_Q \geq \fv(Q)$.

  Consider $Q \subseteq T$ s.t.\ $t \in Q$ and $v(t') < k - 1$ for some $t'$.
  At the step in $\PLTR$ considered by us, we hence have $m_{t'} \geq k - 1 > v(t')$ and therefore before our decrement of $m_t$ we had $m_Q > \vol(Q) \geq \fv(Q)$ which implies $m_Q \geq \fv(Q)$ after the decrement of $m_t$.

  Finally, consider $Q \subseteq T$ s.t.\ $t \in Q$ and $v(t') < k$ for some $t' > t$.
  Again at the step in $\PLTR$ considered by us, we have $m_{t'} \geq k > v(t')$ which implies $m_Q \geq \fv(Q)$ after our decrement of $m_t$.

  If for $t$ no $Q$ exists as characterized in the proposition, $t$ cannot have been the result of $\keepidle$ at this step in $\PLTR$, which is a contradiction.
\end{proof}

\begin{definition}
  We call such $Q$ for activations $t$ of processor $k$ characterized by Lemma~\ref{lemma:critical} \emph{tight set $Q_t$ over activation $t$ of processor $k$}.
\end{definition}

\begin{definition}
  A \emph{critical set $C_t \subseteq T$ over an activation $t$} is the maximum of the set of tight sets $Q_t$ over activation $t$ in regard to the density $\phi$, i.e.
  \begin{align}
    C_t \coloneqq \argmax \{ \phi(Q) \mid Q \subseteq T~\text{is tight set over}~t \} \text{.}
  \end{align}
  As the set of these critical sets $C_t$ for fixed $t$ is closed under union, for the sake of uniqueness, we take $C_t$ to be the inclusion-maximal critical set over activation $t$.
\end{definition}
\subsection{Definitions based on critical sets}
\begin{definition}
  We define a total order $\precsim$ on the set of critical sets $C_t$ over all activations $t$.
  For activations $t, t' \in T$ of processors $k$ and $k'$ respectively, we define $C_t \precsim C_{t'}$ if and only $k < k'$ or $k = k'$ and $t \geq t'$.
  In other words, $\precsim$ is the same order in which $\PLTR$ calculates the activations: from Top-Left to Bottom-Right.
\end{definition}
\begin{definition}
  Let $\rank: \{C_t\} \rightarrow \mathbb{N}$ be a mapping to the natural numbers corresponding to $\precsim$, i.e.
  \begin{align}
    \rank(C_t) \leq \rank(C_{t'})
    \Leftrightarrow
    C_t \precsim C_{t'}
  \end{align}
\end{definition}
\begin{definition}
  Let $\crit: \{C_t\} \rightarrow [m]$ be a mapping to the processors s.t.
  \begin{align}
    \crit(C_t) = c
    \Leftrightarrow
    c~\text{is the highest processor activated at}~t
  \end{align}
\end{definition}
\begin{definition}
  We extend these definitions to general time slots $t \in T$.
  \begin{align}
    \rank(t) \coloneqq
    \begin{cases}
      \max \{\rank(C) \mid t \in C \}
      & \text{if}~t \in C~\text{for some critical set}~C
      \\0
      & \text{otherwise}
    \end{cases}
  \end{align}
  \begin{align}
    \crit(t) \coloneqq
    \begin{cases}
      \max \{\crit(C) \mid t \in C \}
      & \text{if}~t \in C~\text{for some critical set}~C
      \\0
      & \text{otherwise}
    \end{cases}
  \end{align}
  We also extend the definitions to intervals $D \subseteq T$.
  \begin{align}
    \rank(D) &\coloneqq \max \{ \rank(t) \mid t \in D \}
    \\ \crit(D) &\coloneqq \max \{ \rank(t) \mid t \in D \}
  \end{align}
\end{definition}
\begin{definition}
  Let $C$ be a critical set. A nonempty interval $V \subseteq T$ is a \emph{valley of $\rank(C)$} if $C \sim V$ and $V$ is inclusion maximal.
  Let $C_1, \ldots, C_l$ be the (maximal) intervals of $C$.
  A nonempty interval $V$ is a \emph{valley of $C$} if $V$ is exactly the interval between $C_{a}$ and $C_{a+1}$ for some $a < l$, i.e. $V = [\max C_a + 1, \min C_{a+1} - 1]$.
\end{definition}
\begin{definition}
  For a critical set $C$, an interval $D$ \emph{spans} $C$ if $D \cap C$ contains only full subintervals of $C$ and at least one subinterval of $C$.
  The \emph{left valley} $V_l$ of $C$ and an interval $D$ spanning $C$ is the valley of $C$ ending at $\min (C \cap D) - 1$ (if such a valley of $C$ exists).
  The \emph{right valley} $V_r$ of $C$ and an interval $D$ spanning $C$ is the valley of $C$ starting at $\max (C \cap D) + 1$ (if such a valley of $C$ exists).
\end{definition}

\begin{definition}
  For a valley $V$, we define the jobs $J(V) \subseteq J$ as all jobs which are scheduled by $S_{\PLTR}$ in every $t \in V$.
\end{definition}

\begin{lemma}\label{lemma:valley}
  For every critical set $C$ with $c \coloneqq \crit(C)$, every interval $D$ spanning $C$:
  if $\phi(C \cap D) \leq c - \delta$ for some $\delta \in \mathbb{N}$, then $V_l$ or $V_r$ is defined and $|J_{V_l}| + |J_{V_r}| \geq \delta$, where we take $|J_V| \coloneqq 0$ if $V$ does not exist.
\end{lemma}
\todo{Provide a rough visual sketch here}
\begin{proof}
  \todo{Make math more readable in this whole proof, e.g.\ by using fractions and display math or by replacing division by multiplication}
  By choice of $C$ as critical set with $c = \crit(C)$ we have $\vol(C \cap D) \geq (c-1) \cdot |C \cap D|$.
  If this inequality is fulfilled strictly, i.e.\ if
  $\vol(C \cap D) > (c-1) \cdot |C \cap D|$, then with
  $\fv(C \cap D) / |C \cap D| \leq c - \delta$ we directly get $\uv(C \cap D) / |C \cap D| > \delta - 1$.
  This implies that there are at least $\delta$ jobs $j$ scheduled in $C \cap D$ with $\uv(j, C \cap D) > 0$.
  Such jobs must have $E_j \cap (C \setminus D) \neq \emptyset$ and hence at least one of $V_l$ and $V_r$ for $C$ and $D$ must exist and the jobs must be contained in $J_{V_l}$ or $J_{V_r}$.

  If on the other hand we have equality, i.e.\ $\vol(C \cap D) = (c-1) \cdot |C \cap D|$, then let $t$ be the activation of processor $c$ for which $C$ is critical set for.
  Since $\vol(t) > c-1$, we must have $t \notin C \cap D$.
  By the same argument as before, we have that if $\fv(C \cap D) / |C \cap D| \leq c - \delta$, then $\uv(C \cap D) / |C \cap D| \geq \delta + 1$.
  Now suppose that there is no job $j$ scheduled in $C$ s.t.\ $\spac(j, C \cap D) > 0$.
  Then $\fv(C \setminus D) = \vol(C \setminus D) > (c-1) \cdot | C \cap D|$.
  Hence $\fv(C \setminus D) = \vol(C \setminus D) > (c - 1) \cdot |C \cap D|$.
  We then get $\phi(C \setminus D) = \vol(C \setminus D) > (c-1) \cdot (C \cap D)$ since by case assumption $t \in C \setminus D$.
  In conclusion, $C \setminus D$ is still a tight set over $t$ but has higher density than $C$, contradicting the choice of $C$.
  Therefore, there must exist a job $j$ scheduled in $C$ with $\spac(j, C \cap D) > 0$ and hence
  \begin{align}
    \frac
    {\uv(C \cap D) + \spac(j, C \cap D)}
    {|C \cap D|}
    > \delta - 1 \text{,}
  \end{align}
  which again implies that there must be at least $\delta$ jobs scheduled in $C$ with an execution interval intersecting both $C \setminus D$ and $C \cap D$.
  This implies that the left valley $V_l$ or the right valley $V_r$ of $C$ and $D$ exist and that at least $\delta$ jobs are contained in $J_{V_l}$ or $J_{V_r}$.
\end{proof}

\section{Modification of our Schedule}
We modify the schedule $S_{\PLTR}$ returned by our algorithm in two steps.
The first step augments specific processors with auxiliary active slots, s.t.\ in every critical set $C$, there are at least the first $crit(C)$ processors active.
Recall that for the single processor $\LTR$ algorithm, the crucial property for the approximation guarantee was that every idle interval of $S_{\OPT}$ can intersect at most $2$ distinct idle intervals of $S_{\LTR}$.
\todo{Give some high level explanation that we realign the jobs of $J_{V_l}, J_{V_r}$ to higher processors where necessary.}
The second modification step is more involved and establishes this crucial property on every processor $k \in [m]$ by making use of Lemma~\ref{lemma:valley}.
It is important to note that these modification steps are only done for the sake of the analysis.
By making sure that the costs can only be increased by this modification, we get an upper bound for the costs of $S_{\PLTR}$.

\subsection{Augmentation}
We transform $S_{\PLTR}$ into $S_{\aug}$ by adding for every $t$ with $k \coloneqq crit(t) \geq 2$ and $\vol(t) = k-1$ an auxiliary active slot on processor $k$.
This auxiliary active slot does not count towards the volume.

\begin{lemma}\label{lemma:augmented}
  In $S_{\aug}$, in every $t \in T$ with $crit(t) \geq 2$ processors $1, \ldots, crit(t)$ are active.
\end{lemma}
\begin{proof}
  The property directly follows from our choice of the critical sets, the definition of $\crit(t)$ and the construction of $S_{\aug}$.
\end{proof}

\subsection{Realignment}

\begin{algorithm}[H]
  \caption{Realignment of $S_{\aug}$}\label{alg:real}
\begin{algorithmic}
  \State{$\res(V) \gets 2 |J_V|$ for every valley $V$}
  \For{$k \gets m$ \textrm{to} $1$}
    \State{\Call{$\fillop$}{$T$}}
    \State{$\res(V) \gets \res(V) - 1$ for every $V$
    s.t.\ some $V'$ with $V' \cap V \neq \emptyset$ was closed on processor $k$}
  \EndFor{}
  \Function{$\fillop$}{$k, V$}
    \If{$\crit(V) \leq 1$}
      \State{\Return{}}
    \EndIf{}
    \State{let $C$ be critical set s.t.\ $C \sim V$}
    \While{exists active interval $A \subseteq V$ on processor $k$
    with $A \sim V$ and $\hat \phi(A) \leq k - 1$}
      \State{let $V_l, V_r$ be the left and right valley for $C$ and interval $A$ (if $A$ spans $C$)}
      \If{$V_l$ exists and $\res(V_l) > 0$}
        \State{\Call{$\close$}{$k, V_l$}}
      \ElsIf{$V_r$ exists and $\res(V_r) > 0$}
        \State{\Call{$\close$}{$k, V_r$}}
      \EndIf{}
    \EndWhile{}
    \For{every valley $V' \subseteq V$ of $C$ which has not been closed on $k$}
      \State{\Call{$\fillop$}{$k, V'$}}
    \EndFor{}
  \EndFunction{}

  \Function{$\close$}{$k, V$}
    \For{every $t \in V$ which is idle on processor $k$}
      \If{processors $1, \ldots, k-1$ idle at $t$}
        \State{introduce new auxiliary active slot on processor $k$ at time $t$}
      \Else{}
        \State{move active slot at time $t$ of highest processor
        among $1, \ldots, k-1$ to processor $k$ at $t$}
      \EndIf{}
    \EndFor{}
  \EndFunction{}
\end{algorithmic}
\end{algorithm}

\subsection{Invariants for Realignment}
\begin{lemma}\label{lemma:invariant}
  For an arbitrary step during the realignment of $S_{\aug}$ let $k_V$ be the highest processor s.t.\
  \begin{itemize}
    \item
      processor $k_V$ is not fully filled yet, i.e.\ $\fillop(k_V, T)$ has not yet returned,
    \item
      no $V' \supseteq V$ has been closed on $k_V$ so far and
    \item
      there is a (full) active interval $A \subseteq V$ on processor $k_V$.
  \end{itemize}
  We take $k_V \coloneqq 0$ if no such processor exists.
  At every step in realignment of $S_{\aug}$ the following invariants hold.
  \begin{enumerate}
    \item
      If $\phi(C \cap D) \leq k_V - \delta$ for some $\delta \in \mathbb{N}$ and some interval $D \subseteq T$ spanning $C$, then the left or right valley $V_l, V_r$ of $C, D$ exists and $\res(V_l) + \res(V_r) \geq 2 \delta$.
    \item
      For every $t \in C \cap V$, processors $1, \ldots, k_V$ are active at $t$.
    \item
      Every active interval $A \subseteq V$ on processor $k_V$ with $A \sim V$ spans $C$.
  \end{enumerate}
\end{lemma}
\begin{proof}
  We show properties 1 and 2 via structural induction on the realigned schedule $S_{\real}$.
  Then we show that invariant 2 implies invariant 3.
  For the induction base, consider $S_{\aug}$, let $V$ be an arbitrary valley in $S_{\aug}$ and $C$ the critical set with $C \sim V, \crit(V) \coloneqq c$.

  We have $k_V \leq c$, otherwise $V$ contains a full active interval on processor $k_V > c$ and hence also an activation $t \in V$ of processor $k_V$, which by construction of $S_{\aug}$ would have $\crit(t) = k_V > c$. This is a direct contradiction to $\crit(V) = \max_{t \in V} \crit(t) = c$.

  The second invariant now follows since by construction of $S_{\aug}$ and our choice of $C$ we have for every $t \in C$ that processors $1, \ldots, k_V, \ldots, c$ are active at $t$.

  For the first invariant, let $D$ be an interval spanning $C$ with $\phi(C \cap D) \leq k_V - \delta$ for some $\delta \in \mathbb{N}$.
  With $k_V \leq c$ we get $\phi(C \cap D) \leq c - \delta$ and hence by Lemma~\ref{lemma:valley}, we have that the left or right valley $V_l, V_r$ of $C$ and $D$ exist and $|J_{V_l}| + |J_{V_r}| \geq \delta$.
  With the initial definition of $\res(V)$ we get the desired lower bound of $\res(V_l) + \res(V_r) \geq 2 \delta$.

  Now suppose that invariants 1 and 2 hold at all steps of the realignment up to a specific next step.
  Let $V$ again be an arbitrary valley of $\crit(V) \geq 2$ and $k$ the processor currently being filled.
  Let furthermore $k_V, k'_V$ be the critical processor for $V$ before and after, respectively, the next step in the realignment.
  We consider four cases for the next step in the realignment.

  %\begin{enumerate}
  %  \item
  %    Some $V' \supseteq V$ is closed on processor $k$.
  %  \item
  %    Some $V' \subset V$ is closed on processor $k$.
  %  \item
  %    Some $V'$ with $V' \cap V = \emptyset$ is closed on processor $k$.
  %  \item
  %    The call to $\fillop(k, T)$ returns and $\res(V')$ is decreased by 1 for every $V'$ such that some valley intersecting $V'$ has been closed during $\fillop(k, T)$.
  %\end{enumerate}
  \begin{description}
    \item[Case 1:]
      Some $V' \supseteq V$ is closed on processor $k$.
      Then no valley $W$ intersecting $V$ has been closed so far on $k$.
      Also, since $\close$ only moves the active slot highest active processor below $k$, we know that the stair property \todo{define stair property} holds within $V$ on processors $1, \ldots, k$.
      We show that the closing of $V'$ on $k$ reduces the critical processor of $V$ by at least $1$, i.e. $k'_V \leq k_V - 1$.
      If $k_V = k$, then $V' \supseteq V$ is closed on processor $k_V$ and hence by definition we have $k'_V \leq k_V - 1$.
      If $k_V < k$, suppose for contradiction that $k_V \leq k'_V \leq k$, where $k'_V \leq k$ again by definition since $V' \subseteq V$ is closed on processor $k$.

      Let $A \subseteq V$ be a full active interval on $k_V$ before the close of $V'$.
      We show that $A \subset V$, i.e.\ that there must be some $t \in V$ idle on $k_V$ before the close and hence by the stair-property processors $k_V, \ldots, k$ idle at $t$ before the close.
      If $V' \supseteq V$ is closed, clearly $V \subset T$ by the choice of $V_l$ and $V_r$ as valleys of some critical set in the realignment definition.
      Hence we know that $\min V - 1 \in T$ and $\max V + 1 \in T$.
      \todo{Double check if this is true (or if we only have one of the two guaranteed): Since $V$ must be valley \emph{of} some $C$, we should have both $0 \notin V$ and $d^* \notin V$.}
      We show that $t \coloneqq \min V - 1$ must be active on $k_V$ before the close.
      Let $W \supseteq V$ be the valley with $W \sim t$ and $t \in W$.
      We know that $W \supseteq V$ since $W \sim t \succ V$ since $V$ is a valley.
      By our case assumption, no $W' \subseteq W$ can have been closed on processor $k$ so far.
      With $W \supseteq V$ and the definition of $k_W$ we get $k_W \geq k_V$, where $k_W$ is the critical processor of $W$ before the close.
      Our induction hypothesis now implies that processors $1, \ldots, k_V, \ldots, k_W$ are active at $t$ before the close.
      For $A \subseteq V$ to be a (full) active interval on $k_V$ before the close, we hence must have $\min V \notin A$.
      We know by definition of the realignment and function $\close$, that for every $k'$ with $k_V \leq k' < k$ and every $t \in V$:
      \begin{itemize}
        \item
          If $t$ was idle on $k'$ before the close, then $t$ is still idle on $k'$ after the close (definition of $\close$, $k' < k$).
        \item
          If $t$ was idle on $k_V$ before the close, then $t$ was idle on $k'$ before (stair-property) and hence $t$ is still idle on $k'$ after the close.
        \item
          If $t$ was part of full active interval $A \subset V$ on $k_V$ before the close, then $t$ was idle on $k_V + 1$ before the close (choice of $k_V$).
          Hence $t$ was idle on $k$ before (stair-property) and hence $t$ is idle on $k_V$ after the close.
      \end{itemize}
      Taken together, for $t \in V$ to be active on $k'$ after the close, $t$ must have been active on $k'$ before the close (definition $\close, k' < k$) and $t$ cannot have been part of a full active interval $A \subseteq V$.
      Hence $t \in A$ for some \todo{properly define partial active intervals} \emph{partial} active interval $A \subseteq V$ on $k'$ before the close.
      For $A' \subseteq V$ to be a full active interval on $k'_V$ after the close (with $k_V \leq k'_V < k$), we must have $A' \subseteq A$.
      \begin{figure}[H]
        \centering
        \includegraphics[width=\textwidth]{graphics/sketches/invariant_case1.png}\label{fig:invariant_case1}
        \caption{The situation for case $1$ in Lemma~\ref{lemma:invariant}.}
      \end{figure}
      Hence there must have been an active interval $A'' \subseteq [\min A', \max A]$ on processor $k'_V + 1 > k_V$ before the close, which contradicts the definition of $k_V < k$.
      In conclusion, we have $k'_V \leq k_V - 1$ which allows us to prove our invariants 1 and 2.
      If $\phi(C \cap D) \leq k'_{V} - \delta$ for some $\delta \in \mathbb{N}$ and some interval $D$ spanning $C$, then $\phi(C \cap D) \leq k_V - (\delta + 1)$ and hence by induction hypothesis the left or the right valley $V_l, V_r$ for $C, D$ exists and $\res(V_l) + \res(V_r) \leq 2 (\delta + 1)$ both before and after the close.


    \item[Case 2:]
      Some $V' \subset V$ is closed on processor $k$.
    \item[Case 3:]
      Some $V'$ with $V' \cap V = \emptyset$ is closed on processor $k$.
    \item[Case 4:]
      The call to $\fillop(k, T)$ returns and $\res(V')$ is decreased by 1 for every $V'$ such that some valley intersecting $V'$ has been closed during $\fillop(k, T)$.
  \end{description}

\end{proof}



%\bibliographystyle{plain}
\bibliographystyle{abbrvnat}
%\setcitestyle{authoryear,open={(},close={)}}
\bibliography{references}
\end{document}
