\documentclass[a4paper]{article}
\usepackage[margin=1in]{geometry}
\usepackage[utf8]{inputenc}
\usepackage[english]{babel}
%\usepackage{graphicx}
\usepackage{amssymb}
\usepackage{amsmath}
\usepackage{amsthm}
%\usepackage{listings} % Darstellen von Codebeispielen
\usepackage{titlesec}
%\usepackage{chngcntr}
\usepackage{mathtools}
\usepackage{algorithm}
\usepackage{algpseudocode}
\usepackage[normalem]{ulem}
%\usepackage{float}
%\usepackage{cite}
\counterwithin{figure}{section}
\newcommand{\sectionbreak}{\clearpage}
\usepackage[round]{natbib}
\usepackage{float}
\usepackage{svg}
\usepackage{fancyhdr}
\pagestyle{fancy}
\usepackage[hidelinks]{hyperref}

\mathtoolsset{showonlyrefs}

\DeclareMathOperator{\argmax}{argmax}
\DeclareMathOperator{\argmin}{argmin}
%\DeclareMathOperator{\next}{next}
\DeclareMathOperator{\prev}{prev}
\DeclareMathOperator{\on}{on}
\DeclareMathOperator{\off}{off}
\DeclareMathOperator{\idle}{idle}
\DeclareMathOperator{\act}{active}
\DeclareMathOperator{\costs}{costs}
\DeclareMathOperator{\adjacent}{adjacent}
\DeclareMathOperator{\neighboring}{neighboring}
\DeclareMathOperator{\true}{true}
\DeclareMathOperator{\false}{false}
\DeclareMathOperator{\ifop}{if}
\DeclareMathOperator{\elseop}{otherwise}
\DeclareMathOperator{\OPT}{OPT}


\newtheorem{theorem}{Theorem}
\newtheorem{lemma}[theorem]{Lemma}
\newtheorem{definition}[theorem]{Definition}
%\newtheorem*{remark}{Remark}


\title{Greedy Energy-Efficient Scheduling Algorithms for Processor Systems}
%\author{Gunther Bidlingmaier}
%\date{15.11.2020}

\begin{document}

\selectlanguage{english}
%\frontmatter{}
%\input{pages/acknowledgements}

%\maketitle
\begin{abstract}
  We study a particular scheduling setting in which a set of $n$ jobs with individual release times and deadlines has to be scheduled across $m$ homogeneous processors while minimizing the consumed energy.
  Idle processors can be turned off so as to save energy, while turning them on requires a fixed amount of energy.
  For the special case of a single processor, the greedy algorithm Left-to-Right guarantees an approximation factor of $2$.
  We generalize this simple greedy policy to the case of multiple processors and show that the energy costs are still bounded by $2 \OPT + P$.
  Our algorithm has a running time of $(n + m) \log(d^*) F$ where $d^*$ is the largest deadline and $F$ the costs of a maximum flow calculation for checking feasibility of an instance.
\end{abstract}

\tableofcontents

\section{Algorithm}
\begin{algorithm}[H]
\caption{Recursive Formulation of LTR}\label{alg:RECLTR}
\begin{algorithmic}
  \State{} \Return{} \Call{ltr}{$p, 0, \text{empty mapping}$}
  \State{} where
  \Function{ltr}{$q,t,S$}
    \If{$q = 0$}
      \Return$S$
    \ElsIf{$t = D$}
      \Return\Call{ltr}{$q-1,0,S$}
    \Else{}
      \State$\mathcal{T} \coloneqq\Call{keepidle}{q,t,S}$
      \State$S(q,t') \coloneqq \idle$ for $t' \in [t, \mathcal{T})$
      \State$\overline{\mathcal{T}} \coloneqq\Call{keepactive}{q,\mathcal{T},S}$
      \State$S(q,t') \coloneqq \act$ for $t' \in [\mathcal{T}, \overline{\mathcal{T}})$
      \State \Return\Call{ltr}{$q, \overline{\mathcal{T}}, S$}
    \EndIf
  \EndFunction
\end{algorithmic}
\end{algorithm}





%\bibliographystyle{plain}
\bibliographystyle{abbrvnat}
\setcitestyle{authoryear,open={(},close={)}}
\bibliography{references}
\end{document}
